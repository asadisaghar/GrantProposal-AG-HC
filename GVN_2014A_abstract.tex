\documentclass[a4paper, 11pt]{article}
\oddsidemargin=-0.54cm
\evensidemargin=-0.54cm
\topmargin=-1.2cm
\textwidth=17cm
\textheight=25cm
\pagestyle{empty}
    \usepackage[
    top    = 3.0cm,
    bottom = 2.0cm,
    left   = 2.5cm,
    right  = 2.2cm]{geometry}
\begin{document}
The mismatch between the faint-end of galaxy luminosity function and the low-mass end of the dark matter halo mass function has been around for over a decade now. This is a largely-untested challenge for the cold dark matter model. While gravitational lensing provides a unique method to detect the predicted extremely faint or completely dark galactic subhalos, observations of very high angular resolution are required. Previous VLBA maps of the gravitationally-lensed quasar in B1152+199 have resulted in a tentative detection of such a substructure with mass 1e5-1e7 Msolar, based on the jet curvature seen in one of the two macroimages in this system. Here, we propose 3.6 cm observations of B1152+199, using the global VLBI array, providing ~4 times better resolution (0.7 mas) than the 6 cm VLBA data to A) confirm the jet curvature and B) search for previously unresolved distortions in the curved jet to provide the first robust detection of gravitational millilensing by dark halo substructure.

%\section{Big, unresolved problem}
%(1)\\
%The mismatch between the faint-end of galaxy luminosity function and the low-mass end of the dark matter halo mass function has been around for over a decade now. This is a largely-untested issue of the cold dark matter model. 

%\section{Why mission critical?}
%(2)\\
%While gravitational lensing provides a unique method to detect the extremely faint or completely dark galactic subhalos, observations of very high angular resolution are required.  

%\section{Narrow-down} 
%{\bf Focusing on a piece of 1 related to my research}\\
%(3)\\
%Previous VLBA maps of the gravitationally-lensed quasar in B1152+199 have resulted in a tentative detection of such a substructure with mass 1e5-1e7 Msolar, based on the jet bending seen in one of the two macroimages in this system. 

%%Previous VLBA maps of the gravitationally-lensed system B1152+199 have suggested the presence of a dark substructure of mass 1e5-1e7 Msolar within the halo of tha lens galaxy. Here, we are proposing 3.6-cm observations of the two images of the lensed quasar, one of which was found to be anomalously curved, unlike its straight counterpart. While this curvature is barely resolved in the present data (angular resolution ~ 3mas), we expect to achieve 
%\section{Reasoning}
%{\bf ``first time'', ``first detection'', what happened to make this ``first'' possible! Advertise for 1}\\
%If such features are detected, this would constitute the first robust detection of gravitational millilensing.

%\section{Summarizing1}
%{\bf capability, technical advance, idea; more specifically than in 4}\\
%(4)\\
%Here, we propose 3.6 cm observations of B1152+199 with ~4 times better resolution (0.7 mas) than the VLBA data to A) confirm the jet bending and B) search for previously unresolved distortions in the curved jet to provide the first robust detection of gravitational millilensing by dark halo substructure.
%\section{Summarizing2 (Specific aims)}
%{\bf ITEMIZE the specific aims!}\\
%(5)\\
%The goal of the proposed project is to map the B1152+199 system at 8.4GHz with $\approx 0.7$ mas resolution to A) confirm the jet bending seen in the Rusin et al. (2002) 5GHz VLBA data and B) search for previously unresolved distortions. If such features are detected, this would constitute the first robust detection of gravitational millilensing. 

%\section{How to evaluate}
%{\bf How do I know if I have reached my aims in 6; Real SPECIFIC!}\\

%\section{Concluding}
%{\bf couple back to 3, 2, 1}\\

\end{document}


%A generic, but largely untested prediction of the cold dark matter model is that the dark halos of galaxies should contain large numbers of extremely faint or completely dark subhalos. Previous VLBA maps of the gravitationally lens B1152+199 have resulted in a tentative detection of such a substructure with mass 1e5-1e7 Msolar, based on the jet bending seen in one of the two macroimages in this system. Here, we propose EVN/1.3 cm observations of B1152+199 with ~10 times better resolution (0.3 mas) than the VLBA data to A) confirm the jet bending and B) search for previously unresolved distortions in the curved jet to provide the first robust detection of gravitational millilensing by dark halo substructure.

