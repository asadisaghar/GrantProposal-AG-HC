% Example LaTeX commands to generate a NorthStar justification ps/pdf file
\documentclass[a4paper, 11pt]{article}
\usepackage{graphics,graphicx}
\oddsidemargin=-0.54cm
\evensidemargin=-0.54cm
\topmargin=-1.2cm
\textwidth=17cm
\textheight=25cm
\pagestyle{empty}
    \usepackage[
    top    = 3.0cm,
    bottom = 2.0cm,
    left   = 2.5cm,
    right  = 2.2cm]{geometry}
\begin{document}

\section{Abstract}
Astronomical observations have shown that black holes exist at two different mass scales -- stellar--mass black holes formed from the collapse of dying stars, and so--called {\it supermassive} black holes fed by the accretion of gas into the central regions of galaxies. While theoretical mechanisms for the formation of black holes at intermediate mass scales have been proposed, the empirical evidence for such objects has remained scant. However, the potential of exploiting effects due to gravitational lensing – the bending of light by strong gravitational fields – to hunt these objects down has so far been largely unexplored.
Gravitational lensing has already allowed astronomers to find planets outside our solar systems, to estimate the masses of galaxies and probe the dark matter of the Universe. We have singled out a case of such systems compatible with our simulations of gravitationally lensed radio jets and were granted 12 hours of time with the global VLBI array to test our hypothesis. Previous VLBA maps of the gravitationally-lensed quasar in B1152+199 have resulted in a tentative detection of a dark compact substructure in the main lens with mass 10$^5$--10$^7$ M$_\odot$, based on the jet curvature seen in one of the two macroimages in this system. We are now in posession of a new set of 3.6 cm observations of the system B1152+199, using the global VLBI array, providing $\sim$4 times better resolution (0.7 milliarcsecond) than the 6 cm VLBA data to A) confirm the jet curvature and B) search for previously unresolved distortions in the curved jet to provide the first robust detection of gravitational millilensing by dark halo substructure. Our numerical models suggest that if the peculiar appearance of this radio source is indeed due to gravitational lensing by a foreground object, then that object must have properties very similar to an intermediate-mass black hole. If confirmed, this would not only be the smallest gravitational lens ever discovered, but also the first case of a black hole discovered through gravitational lensing, at any mass scale.

\section{Gravitational lensing and black holes}
Rays of light do not always follow straight paths. Gravitational lensing is a well-known effect in astronomy, by which overdensities of matter along the line of sight cause a bending in the light from distant light sources. The size of this effect was accurately predicted by Einstein in 1915, and has in the last few decades helped astronomers to detect free-floating exoplanets, to estimate the masses of galaxies and galaxy clusters, to measure cosmological parameters and constrain the nature of dark matter (for a review, see Bartelmann 2010). Secondary gravitaional lensing effect is also used to probe over densitied inside the primary lens (i,e. galaxies in galaxy clusters or dwarf galaxies in galactic lens systems). Figure 1 features a schematic example of a secondary gravitational lensing effect, in which an overdensity inside the main lens (the galaxy) happens to lie in the line of sight of one of the images made by the main lens. The ovserdensity causes a small-scale perturbation in the spacetime athe the position of one of the images. This distors that lensed image of the source without affecting the other images which also makes it possible to distinguish the secondary lensing effect from the interinsic structure of the source.

\begin{figure}[tbh]
\centering
\includegraphics[scale=0.3]{Figure-lensing-v3.jpg}
\caption{Schematic illustration of gravitational millilensing as a probe of the local overdensity (an interdediate-mass black hole or a dark matter substructure). A foreground galaxy with a dark matter halo produces two images of a background light source (macroimages). Am intermediate-mass black hole located in the dark halo intercepts the path of one of these macroimages and produces a small-scale distortion (millilensing) in its surface brightness distribution. Whereas morphological anomalies intrinsic to the source should be mimicked in both macroimages, millilensing will affect each macroimage differently, and typically turn up just in one image.
}
\end{figure}

The angular scale of the distortion depicted in Figure 1 is primarily determined by the mass of the secondary lens (the black hole). A galaxy-mass lens gives rise to a ring with a radius of order one arcsecond (1/3600 of a degree). Objects at intermediate--mass black hole mass scale are expected to give rise to lensing effects on smaller angular sizes, and current radio interferometers are in principle able to detect resolved ring structures down to 0.1--1 milliarcsecond scales. Even so, secondary lensing effects of the type depicted in Figure 1 made by IMBHs have – until now – not been confirmed. The proposed project revolves around a potential detection of a curvature in one of the images of an already lensed quasar jet -- the very first case of its kind. Lensing effects at this scale can be produced either by dark matter structures in the dwarf--galaxy mass range or by intermediate--mass black holes (e.g. Zackrisson \& Riehm 2010, Zackrisson et al. 2013), but only objects in the latter category are sufficiently compact to produce distinct lensing features on radio jets (Zackrisson et al. 2013). If our interpretation is correct, this would then make our target the first--ever detection of an intermediate--mass black hole through gravitational lensing. Astronomers already have strong observational evidence for the existence of black holes at two different mass scales (for a review, see Narayan \& McClintock 2013): Stellar--mass black holes (5--30 times the mass of the Sun) and supermassive black holes ($\sim$ 10$^6$ -- 10$^9$ times the mass of the Sun). It has been postulated that intermediate--mass black holes ($\sim$ 10$^2$ -- 10$^6$ times the mass of the Sun) could form from the collapse of the central regions of star clusters (e.g. Portegies Zwart et al. 2004), from the collapse of very massive stars (e.g. Freese et al. 2010) or from direct collapse of gas clouds in the early Universe (e.g. Yue et al. 2014), but the empirical evidence for such black holes remains controversial (see Kormendy \& Ho 2013 for a review). If the mass of the object responsible for the lensing in our target object can be accurately pinned down to lie in the $\sim$ 10$^2$ -- 10$^6$ Solar mass range, this would hence have important implications for our understanding of the cosmic mass distribution of black holes.

\section{The case of secondary gravitational lensing at milliarcsecond scales }
Our team has had a long--vested interest in hunting down cases of gravitational lensing on milliarcsecond scales (Zackrisson et al. 2008, Riehm, Zackrisson et al. 2009, Zackrisson \& Riehm 2010, Zackrisson et al. 2013). By going through archival data from various high--resolution radio interferometers, we have come across a candidate for such small--scale gravitational lensing. The object in question, B1152+199, is a strong lensing system, discovered as part of the Cosmic Lens All--Sky Survey (CLASS), consisting of a quasar's radio jet at z = 1.019 lensed by a single galaxy at z = 0.439 into two images which are 1.56" apart in the sky (Myers et al. 1999). The single--lens model of the system, based on 5GHz VLBA maps of the blazar as well as {\it I}-- and {\it V}--band HST images revealing the lens galaxy (Rusin et al. 2002), was shown to be insufficient to explain the anomalous curvature in one of the images absent in the other (Figure 3). Metcalf (2002) suggested that the curvature in image B is not an intrinsic feature of the source, but rather due to one (or more) compact perturber(s) of $M\sim10^5$--$10^7 h^{-1} M_\odot$ on the lens plane and along the line of sight of image B. However, the resolution of the data at 5\,GHz ($\sim3$\,mas where image B is only $\sim15$\,mas long) barely allows further constraints on the mass and inner structure of the perturber(s).


\section{Project description}

\subsection{Scientific impact}
If B1152+199 could be established as a case of secondary gravitational lensing, this would by itself be Nature/Science material. The fact that this could also be the first--ever case of an intermediate--mass black hole detected through gravitational lensing makes the case even more tantalizing. However, to make the case that the curved jet structure of this object is indeed caused by the bending of light by a massive foreground object requires dedicated gravational lens modelling, which we are hereby attempting to secure funding for.

\subsection{Data and interpretation}
The B1152+199 radio source belongs to a class of objects known as active galactic nuclei, which at radio frequencies -- in the absence of gravitaional lensing -- typically appear as bright cores with one or two jets emerging from their central regions. This object has also been observed by powerful radio arrays at four different frequency bands covering 1.4\,GHz to 10.5\,GHz, as well as the Hubble Space Telescope (Rusin et al. 2002). Already in 2001, the unusual curvature in one of the images of the system was noted by Metcalf 2001, and Rusin et al. 2002. They, however, admit that certain conclusions about the porisiont and nature of the potential secondary lens in the system would only be possible if higher resolution images of the system were available. Now, we have access to an archival dataset of the system with a time difference of about 10 years. This is enough time difference to rule out intrinsic source variabilities for such a system with significant variation over time scales of a few months. The preliminary analysis of the archival 5\,GHz data set indicates a persisting curvature in image B (Figure 2). Therefore, given the 10-year time difference between the two data sets and the fact that morphological anomalies produced by millilensing of halo substructures in the $\geq 10^4 M_\odot$ mass range remain stationary over hundreds to thousands of years, the (milli)lensing nature of the jet curvature seems to be confirmed. Our group also made an attempt to observe the same target at two higher frequency bands with the EVN, improving the angular resolution by a factor of $\sim$ 9. Preliminary analysis of all three datasets of the system -- with 2 different frequncy bands covering a period of $\sim$ 15 years -- persistently show an anomaly in the system. Comparing the results of our previous computer simulations of similar systems with the observational evidence, we expect to be able to constrain the position and mass of the potential IMBH in this system.

\begin{figure}[tbh]
\centering
\includegraphics[scale=0.5]{figure2tmp.jpg}
\caption{{\bf  BLAHJBLAHBLAH}The VLBA 5-GHz maps (Rusin et al. 2002; beam $3.6 \times 1.9$ mas$^2$) of the two macroimages of B1152+199, overlaid with lens models from Metcalf (2002). The slight curvature of the jet in the right panel (image B) is attributed to millilensing by a substructure located close to this macroimage. In the absence of such substructure, the jet in the right macroimage would follow the path given by the dotted, straight line (a poor fit to the data). The two curved paths are produced when different forms of substructures are introduced. The lowermost curve is given by a point-mass substructure (IMBH) of mass $\sim 10^5$–-$10^7 M_\odot$ located at the position of the triangle. The intermediate curve is reproduced by two SIS substructures (squares) with poorly constrained masses. Metcalf (2002) emphasizes that while the jet bending indicates some form of millilensing, the substructure solution (in terms of substructure position and mass) to these data is by no means unique.}
\end{figure}


\subsection{Project timeline}
We estimate that it would take about 4 months of full--time work to apply our computer-- based model of gravitational lensing (Zackrisson et al. 2013) to confirm or falsify the gravitational lensing interpretation of B1152+199 by an intermediate--mass black hole or another compact object in the same mass range. I am planning to finish this project that started during my PhD, after my PhD.


\subsection{Team}
Our team consists of:
\begin{itemize}
\item Saghar Asadi (4$^\mathrm{th}$ year PhD student at Stockholm Univeristy), with project title ``Gravitational lensing and radio interferometry as a probe of the small--scale structure of dark matter''. She has worked on simulations of gravitationally--lensed objects with secondary lens effects due to small--scale structures of dark matter with radio interferometers such as the Atacama Large Millimeter/sub--millimeter Array (ALMA).
\item Erik Zackrisson (Associate Professor at Uppsala University), with ample experience in the field of gravitational lens modelling. Erik Zackrisson was recruited by Uppsala University in 2015 to take over leadership of the Galaxies and Cosmology research group at the Department of Physics and Astronomy after the recent retirement of Professor Nils Bergvall.
\end{itemize}
\section*{References}
- Bartelman, M. 2010, Gravitational lensing, Classical and Quantum Gravity, Volume 27, 233001
- Dodson, R., et al. 2008, The VSOP 5 GHz Active Galactic Nucleus Survey. V. Imaging Results for the Remaining 140 Sources, Astrophysical Journal Supplement Series 175, 314
- Freese, K., et al. 2010, Supermassive Dark Stars: Detectable in JWST, Astrophysical Journal, 716, 1397
- Kormendy, J. Ho, L. C. 2013, Coevolution (Or Not) of Supermassive Black Holes and Host Galaxies, Annual Review of Astronomy and Astrophysics, 51, 511
- Lister, M. L., Homan, D. C. 2005, MOJAVE: Monitoring of Jets in Active Galactic Nuclei with VLBA Experiments. I. First-Epoch 15 GHz Linear Polarization Images, Astronomical Journal, 130, 1389
- Narayan, R., McClintock, J. E. 2013, Observational Evidence for Black Holes, in General Relativity and Gravitation: A Centennial Perspective", Editors: A. Ashtekar, B. Berger, J. Isenberg and M.A.H. MacCallum, Cambridge University Press (open access link: http://arxiv.org/abs/1312.6698)
- Portegies Zwart, S. F., et al. 2004, Formation of massive black holes through runaway collisions in dense young star clusters, Nature, 428, 724
- Riehm, T., Zackrisson, E., et al. 2009, Strong Lensing by Subhalos in the Dwarf-galaxy-mass
- Range. II. Detection Probabilities, Astrophysical Journal 700, 1552
- Yue, B., et al. 2014, The brief era of direct collapse black hole formation, Monthly Notices of the Royal Astronomical Society, 440, 1263
- Zackrisson, E. et al. 2008, Strong Lensing by Subhalos in the Dwarf Galaxy Mass Range. I. Image Separations, Astrophysical Journal 684, 804
- Zackrisson, E. \& Riehm, T. 2010, Gravitational Lensing as a Probe of Cold Dark Matter Subhalos (invited review), Advances in Astronomy, 478910
- Zackrisson, E., et al. 2013, Hunting for dark halo substructure using submilliarcsecond-scale observations of macrolensed radio jets, Monthly Notices of the Royal Astronomical Society 431, 2172
\end{document}
